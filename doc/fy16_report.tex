%\documentclass[review]{elsarticle}
\documentclass[preprint,3p]{elsarticle}

\usepackage{hyperref}
\usepackage{xspace}
\usepackage{tabularx}
\usepackage{array}
\usepackage{url}
\usepackage{fancyvrb}
\usepackage{listings}
\DefineShortVerb{\|}
%\usepackage{lineno}
%\modulolinenumbers[5]

\journal{Journal of Unpublished HAD Reports}

%%%%%%%%%%%%%%%%%%%%%%%
%% Elsevier bibliography styles
%%%%%%%%%%%%%%%%%%%%%%%
%% To change the style, put a % in front of the second line of the current style and
%% remove the % from the second line of the style you would like to use.
%%%%%%%%%%%%%%%%%%%%%%%

%% Numbered
%\bibliographystyle{model1-num-names}

%% Numbered without titles
%\bibliographystyle{model1a-num-names}

%% Harvard
%\bibliographystyle{model2-names.bst}\biboptions{authoryear}

%% Vancouver numbered
%\usepackage{numcompress}\bibliographystyle{model3-num-names}

%% Vancouver name/year
%\usepackage{numcompress}\bibliographystyle{model4-names}\biboptions{authoryear}

%% APA style
%\bibliographystyle{model5-names}\biboptions{authoryear}

%% AMA style
%\usepackage{numcompress}\bibliographystyle{model6-num-names}

%% `Elsevier LaTeX' style
\bibliographystyle{elsarticle-num}
%%%%%%%%%%%%%%%%%%%%%%%

\begin{document}

\begin{frontmatter}

\title{Implementing DANE Email Testers}
% \tnotetext[mytitlenote]{Fully documented templates are available in the elsarticle package 
% on \href{http://www.ctan.org/tex-archive/macros/latex/contrib/elsarticle}{CTAN}.}

%% Group authors per affiliation:
\author{Simson Garfinkel}

\begin{abstract}
DNS-Based Authentication of Named Entities (DANE) can be used to
improve the trustworthiness of email by providing a discovery
mechanism for certificates and public keys that are used to secure the  SMTP communications
between Mail Transfer Agents (MTAs) and to encrypt email. A testing
frameworks for each of these DANE applications was created. The SMTP
tester uses a conventional logic-driven approach for test
frameworks, while the OPENPGPKeys and SMIMEA testers employ a
database-driven testing framework that implements the tester as a
state machine. The two approaches are compared in terms of
development time, functionality, and extensibility.
\end{abstract}

\begin{keyword}
DNS-Based Authentication of Named Entities\sep DANE\sep OPENPGPKeys\sep SMIMEA\sep 
\end{keyword}

\end{frontmatter}

%\linenumbers

\section{Introduction}

Public key discovery and certification has been one of the primary
barriers to realizing the benefit of email security
protocols. 

Although many SMTP servers now implement the STARTTLS SMTP
command\cite{rfc3207}, Mail Transfer Agents (MTAs) typically
certificates signed by \emph{any} Certificate Authority (CA)
(including self-signed certificates), as SMTP servers generally lack a
list of pre-configured certificate authorities and a user interface
that could be used to specify the disposition of mail that is being
sent to a mail server that has a certificate that is signed by an
unknown CA. Since mail is sent by default without encryption, this
so-called ``opportunitist encryption'' at least provides security
against a passive monitoring attacker.

After more than two decades of research and standards development
(e.g. \cite{rfc1421,rfc4880,rfc5750}), two incompatiable standards
have emerged for the exchange of encrypted Internet email: S/MIME and
PGP. Both of these standards from the \emph{key discovery problem:} in
order to send a recipient an encrypted message, it is necessary to
first obtain the recipient's public key. Within a single organization,
this problem can be solved through the use of a enterprise
directory. S/MIME also has the advantage that digitally signed email
messages typically include the sender's public key in the PKCS\#7
attachment\cite{rfc2315}, allowing the recipient of a digitally signed
message to send an encrypted reply to the sender. However, when
sending email between organizations, there is currently no deployed
mechanism that can be used to find the recipient's S/MIME certificate
or PGP public key prior to the first message exchange.

DNS-Based Authentication of Named
Entities\cite{rfc7671} (DANE) provides a general mechanism for the
discovery of cryptographic keys associated with DNS-named. By relying
on the DNSSEC\cite{rfc3833} trust roots, DANE delegates control of the
names within an organization's DNS hierarchy to the administrative
unit that is in control of the organization's DNS. RFC7671 and RFC7672 provide a
discovery mechanism for a domain's certified SMTP MTA public key or
PEM certificate\cite{rfc7671,rfc7672}. Work is in progress on a mechanism for discovering
S/MIME certificates associated with email addresses
(|draft-ietf-dane-smime11|), and 
an experimental RFC was recently approved for associating OpenPGPKey public keys with email addresses\cite{rfc7929}.

We created a conventional tester for HTTP and SMTP servers in
Python. The tester consists of a python module that implements an
instrumented client that uses \texttt{getdns}\cite{getdns} to acquire a TLSA RR for either
an HTTP or SMTP server. The certificate is then used to verify a TLS
server reached through either HTTP or SMTP. The tester can be run from
command line or a web interface. Testing results are incorporated into
a data structure which can then be rendered as either text or HTML.

The conventional tester was implemented using commonly accepted
programming practices including object-oriented design,
modularization, and integrated tests. Nevertheless, it proved
difficult to develop, audit, and extend to new protocols. A particular
complication is the fact that whether a DANE validation succeeds or
fails depends on both the server's certificate and the results of DNS
queries, but DNS servers frequently give different answers to the same
questions as a result of caching, load balancing, network congestion,
and configuration changes. Thus, it is difficult to determine the
reason for code that suddenly starts or stops working.

Following the development of the conventional tester, we created a
tester for the draft SMIMEA and OpenPGP protocols. As a result of the
experience with the conventional HTTP and SMTP testers, we developed a
new tester framework based on the core idea of \emph{auditability},
which we define as ``the degree to which transactions can be traced
and audited through a system.'' The core design goal of the SMIME and
OpenPGP tester was to maintain a log of all DNS queries and responses
so that the tester's edicts could be reviewed and validated. Another
design goal was to enable a \emph{replay} capability, so that previous
queries and responses to be used for regression testing. In the
process of designing the new testing framework, we realized the entire
testing agent could be implemented as a state machine, with each phase
of the test corresponding to states and state transitions being
directed by the responses from the systems undergoing test.


\subsection{Availability}

The complete DANE tester is available from Github at \url{https://github.com/usnistgov/dane_tester}.

\section{HTTP and SMTP Server TLSA Tester}
We created a tool to test server certificates advertised using DANE
for validity. Our tester tests both HTTP and SMTP servers. 

\subsection{Overview}

The tool's function is compare the certificates provided by the
SSL server(s) associated with a webserver or email server are
consistent with the records provided by one or more DNSSEC TLSA
RRs. It does so by following these rough steps. 

\begin{enumerate}
\item The user specifies a service to test (either an HTTPS URL or an
  email address).
\item Using getdns, the tester obtains the DANE announcements for the
  service. 
\item The tester connects to the service and reports if the server's
  public key is validated by any of the   certificates or public keys
  advertised through DNS TLSA records. 
\end{enumerate}

To perform these tests, the tester contains its own DANE
implementation. The implementation covers several specific cases,
including:

\begin{itemize}
\item If the DNS name maps to multiple IP addresses, each IP address
  is tested. (Different IP addresses may have different certificates.)
\item If the DNS name specifies a CNAME, the CNAMEs are followed until
  the end of the change. A DANE record is looked for both the original
  DNS name and the name pointed to be the CNAME chain.
\item If the mail delivery host specifies a mail exchanger (MX)
  record, then both the primary name and the exchangers are tested.
\end{itemize}

\subsection{System Requirements}

\paragraph{OpenSSL v 1.0.2h}
The TLSA tester tests certificate validity using the \verb+openssl+
command line program, which is called using Python's \verb+subprocess.Popen()+
facility. Testing requires the flags |-purpose|, |-trusted_first|,
|-partial_chain| and |-CAfile|. These flags are only available on
OpenSSL version 1.0.2h and above.

Because US Government deployments are likely to have the
most recent FIPS-140 accredited OpenSSL version installed, which is
currently OpenSSL version 0.9.8zh, the installing scripts download and
install a private copy of OpenSSL version 1.0.2h in the
\verb+/usr/local/ssl+ directory.

\paragraph{Python 2.7}
The TLSA tester also required that Python modules \verb+getdns+ and
\verb+M2Crypto+ be installed. The only version of Python that supports
both is Python 2.7, so these modules are added to the Python 2.7
installation, even if a version of Python 3 is installed.

\paragraph{M2Crypto} The Python package M2Crypto is used for some
certificate manipulations.

\paragraph{getdns} The TLSA tester uses getdns and Verisign's getdns Python
bindings. Numerous problems resulted from the decision to use getdns
and it was not used in the follow-on tester.

\paragraph{py.test} The Python test framework \texttt{py.test} is used
for unit tests. Be sure to use version Python 2.7.

\subsection{File Layout}

The tester is contained entirely in the {\tt server/} directory of the
git repository. It is implemented as a Python module and a CGI
script. The tester consists of the following shown in Table~\ref{tlsa}.

\begin{table*}
\begin{tabularx}{\textwidth}{>{\tt}lX}
file & purpose\\
\hline
Makefile & Targets for configuring Centos 7 base system and compiling helper programs. \\
README.md & Documentation \\
ca-bundle.crt & File containing multiple PEM-encoded SSL certificates of ``Well known'' Certificate Authorities. Used for Certificate Usage 0, which validates against system trust anchors. \\
centos7-setup.bash & Script to configure Centos 7 system to run the tester. \\
dane\_check.cgi & CGI script to run the tester from a webserver.\\
dane\_checker.py & Python module that implements DANE tester. Called by dane\_check.cgi and runnable directly from the command line. Can output both TEXT and HTML data.\\
dnstree.py & A program to test the getdns system.\\
get\_altnames & The compiled executable for the \emph{get\_altnames.c} program.\\
get\_altnames.c & M2Crypto provides no mechanism to read the alternative names on the X509 certificate, so this helper program gets the alt names of a certificate provided on the command line and prints the names to stdout.\\
test\_dane.py & A test program using the py.test framework to test functions within the {\tt dane\_checker} module.\\
\end{tabularx}
\caption{Files in the \texttt{server/} directory}\label{tlsa}
\end{table*}

\subsection{System Preparation}

Preparing the system for the DANE server tester is complicated by the
tester's use of getdns. The version of getdns that is distributed with
Centos7 has a bug that prevents the verification of DNSSEC signatures
on CNAME RRs. Unfortunately, the version of getdns on the download
site does not compile with the version of Unbound that is in the
Centos7 ``yum'' package repositories. Therefore it is necessary to
download and install unbound and getdns. A further complication is
that the current version of getdns (getdns-1.0.0b2) requires the
\#include file |unbound-event.h|, which the unbound installer does not
install by default. 

Another complication is that the server tester requires a 
version of OpenSSL more recent than 1.0.2h. However, we could not get
getdns to compile with this version. Thus it is necessary to install
unbound and getdns, and then install OpenSSL 1.0.2h in a non-standard
directory (we chose the default installation directory,
|/usr/local/ssl|). 

The installation program |centos7-setup.bash|
addresses some of these issues, but we are not coding around all of
them in the hope that the unbound and getdns distributions will be fixed.
\begin{enumerate}
\item Uncomment \texttt{AddHandler cgi-script .cgi} in the webserver's
  configuration file.
\item Add \texttt{ExecCGI} to the \texttt{Options} directive in the
  webserver configuration file.
\item Run the |centos7-setup.bash| script.
\item Set \verb|$DANESRC| to root of the cloned git repository.
\item Run the script |$DANESRC/server/centos7-setup.bash| %$ to install 
  the major Centos library modules that you need.
\item The script will stop. 
\item Download and install the current versions of unbound. Install
  the additional header directory |unbound-event.h|.
\item Download and install getdns.
\item Rerun the script |$DANESRC/server/centos7-setup.bash| %$
  to complete the installation.
\end{enumerate}

\emph{Note: On the github forum, the maintainers of getdns assert that
getdns should compile with the stock Centos7 unbound any version of
OpenSSL, and further state that the OpenSSL should be specified with
the \texttt{--with-ssl=} configuration option, rather than using the
\texttt{-I} and \texttt{-L} compiler flags to specify the location for the include
files and the library, respectively. This assertion has not been tested.}

\subsection{Installation and Operation}

The DANE server tester can be run from the command line using the
|dane_checker.py| script or from a web browser by placing the
|dane_check.cgi| script and the associated files in a
CGI-enabled directory of a modern web server. By design, the entire
GIT repository can be cloned in the directory of the web server.

\begin{enumerate}
\item Set \verb|$DANESRC| to root of the cloned git repository
\item \verb|cd $DANESRC|
\item Run |make| to compile |get_altname|.
\item Run |py.test| to run the self-test.
\item Run |python2.7 dane_checker.py https://good.dane.verisignlabs.com| to test Verisign's known-good
  DANE https server.
\item Run |python2.7 dane_checker.py https://bad-sig.dane.verisignlabs.com| to test one of Verisign's
  known-bad DANE https servers.
\item Run |python2.7 dane_checker.py --test| to test all of the test vectors.
\item Access \texttt{https://\emph{host}/\emph{directory}/dane\_check.cgi} to run
  the web-based tester
\end{enumerate}


\subsection{Theory of Operation}

To test a DANE server, the Python file |dane_checker.py| contains a
DANE implementation in Python. The test is coded as a series of
imperative statements that determine the name of the remote server,
perform DNS lookups to discover the IP address(es) on which the server
is running, fetch the server certificates, and then compare the server
certificates to the DNS announcements. 

Rather than printing out the results of the tests as they happen, the
testing infrastructure creates a data structure that contains a record
of each test that was performed and the results of the test. The
data structure can then be rendered as text or HTML as required by the
caller. 

The tests were developed by a
close reading of the DANE specifications. Each test is given a name
and a number. Each test description is then stored its own Python
object. The slots of the object are shown in Figure~\ref{slots}; the
test numbers ranges are shown in Table~\ref{ranges}.
\begin{table}
\centering\fbox{\begin{tabularx}{\textwidth}{llX}
slot & type & description \\
num & int & Test number, assigned by the author.\\
desc & str & A description of the test, in English.\\
section & str & The section of the DANE RFC from which the test derives.\\
failed\_desc & str & What to print if the test fails.\\
recommendation & boolean & If the test is for a recommended usage, but not a mandatory usage.\\
\end{tabularx}}
\caption{Slots in the DaneTest object}\label{slots}
\end{table}

\begin{table}
\centering\fbox{\begin{tabular}{ll}
range & meaning \\
\hline
101-104 & DNS tests \\
201-205 & Server verification\\
301-308 & Certificate verification\\
401-407 & Ensemble results\\
\end{tabular}}
\caption{Test numbers for DANE tests}\label{ranges}
\end{table}

When |dane_checker.py| loads the \emph{DaneTest} objects are created
and stored in an associative array.

As the test program executes, traditional imperative logic implements
the DANE client and tests the server for correct responses. Whereas a
traditional tester might simply print the results, this tester instead
creates a list of \emph{DaneTestResult} objects. This list gets longer
as the test progresses towards completion. Most of the tester's
functions return a list of \emph{DaneTestResult} objects, which are
then concatenated to form the final result.

The \emph{DaneTestResult} object is a data structure that contains
structured information about whether each test passes or fails. The
slots of the object are shown in table~\ref{testResults}.

\begin{table}
\centering\fbox{\begin{tabularx}{\textwidth}{llX}
slot & type & description \\
\hline
passed & boolean & True if the test passed \\
dnssec & boolean & True if the test was a DNS lookup and the result was secured with DNSSEC \\
dnsrelied & boolean & True if the DNS lookup was relied upon. (Some DNS lookups may not be relied upon. For example, a DNS name may map to two IP addresses, but only one may have a functioning server.)\\
hostname & str & The relevant hostname.\\
ipaddr & str & The relevant IP address \\
test & DaneTestResult & A reference to the (immutable) \emph{DaneTestResult} object that corresponds to the test.\\
what & str & The context of what was being tested when the test took place.\\
rdata & str & The DNS response data (which the getdns API calls rdata) that was used in the test, if the test involves a DNS query.\\
key & str & Additional information returned in a DNS query, such as an
MX preference or a DANE cryptographic key. \\
\end{tabularx}}
\caption{DaneTestResults Object}\label{testResults}
\end{table}

How this works is illustrated in this snippet of code from the
function \emph{get\_dns\_ip} which is executed after an DNS query for
an MX RR, shown in Figure~\ref{snippet}.

\begin{figure}
\begin{lstlisting}[basicstyle=\small,language=Python]
if a['type'] == getdns.RRTYPE_MX == request_type:
    what = 'DNS MX lookup {} = {} {}'.format(hostname,rdata['preference'],rdata['exchange'])
    ret.append( DaneTestResult(passed=SUCCESS,
                               what=what,
                               dnssec=reply.get('dnssec_status'),
                               rdata=rdata,
                               data=rdata['exchange'],
                               key=rdata['preference']))
\end{lstlisting}
\caption{}\label{snippet}
\end{figure}

In this example, \emph{ret} is an array of \emph{DaneTestResult}
objects that will be returned. If an MX record is returned in the DNS
lookup, it's fields a broken out and added to \emph{ret[]}. DNS can
return multiple RRs in a query, and all of them are stored in
\emph{ret[]}.  They can then be both used in later parts of the test
and rendered for final output.

\subsection{Demo}
The tester is installed on the server dane-test.had.dnsops.gov. The
web page \url{http://dane-test.had.dnsops.gov/dane_tester/server/} is installed in the
directory |/var/www/html/dane_tester/server|.

\section{SMIMEA and OpenPGPKey Database-Driven Tester}
In addition to storing trust information for SSL servers, DANE now
supports the storing of S/MIME certificates and OpenPGP and public keys. This use
of DANE makes it possible for a DANE-aware Mail User Agent (MUA) to
fetch certificate or public key associated with an email address,
making it possible to send encrypted email and to verify digitally
signed mail without the need to use a Public Key Infrastructure (PKI)
or to obtain certificates through a manual process.

We created a tester for SMIMEA and OpenPGPKey protocols. The goal of
these tests was to implement an end-to-end test of mail systems that
use SMIMEA and OpenPGPKey. Thus, in addition to obtaining and verify
keys, the testers needed to be able to send encrypted mail and to
receive and verify digitally signed mail. In total, the 
SMIMEA and OpenPGPKey testers require the following functionality:

\begin{itemize}
\item Finding and fetching the S/MIME certificate or OpenPGP public
  key associated with an email address using the SMIMEA or OpenPGPKey
  protocol.
\item Encrypting a message using S/MIME or PGP.
\item Sending email messages that are not encrypted or encrypted.
\item Automatically processing incoming email messages.
\item Determining if an email message is signed and verifying the
  corresponding signature. 
\end{itemize}

\noindent Based on our experience in developing and using the server tester, we
had additional requirements on the test infrastructure:

\begin{itemize}
\item Track each test, including when the test started, completed, and
  all activity (e.g. DNS queries) associated with each test.
\item Capture the full contents of all DNS queries, so that they could
  be evaluated at a later point in time.
\item Maintain statistics of tester activity, and easily generate the
  statistics on demand.
\end{itemize}

\noindent Given these requirements, we decided to create a new framework for the
tester:

\begin{itemize}
\item Instead of controlling the test progression with a series of
  declarative statements, the new tester would implement each
  step in the test as state in a finite state machine. 
\item Each DNS query would be recorded and logged in the database,
  which would record the time the query was made, the query sent to
  the server, and the server's response. 
\item The single testing infrastructure should support multiple \emph{test
  types}. The database should record all of the steps associated with
  every invocation of a test type by a user, which we call a
  \emph{test instance}.
\item Each test type is given a unique name and identifier. The test type
  consists of a sequence of steps.
\item Each test instance is given a unique identifier. 
\item A \emph{workqueue} keeps track of the state of each running test instance.
\item Because this tester can send email, the tester requires that
  users register to prevent abuse. Registrations are stored in the
  database. Users are identified by email addresses and authenticated
  with a randomly generated password (called a \emph{hash} for
  historical reasons, even though it is not a hash). 
\end{itemize}

\subsection{Overview}
The database-driven tester is implemented with an object-oriented
Python framework, a set of command-line programs and CGI scripts, and
a MySQL database. Currently SQL statements are embedded directly in
the code, although the system could be modified to use an object
relation mapper (ORM) framework. 

\subsection{System Requirements}

\paragraph{OpenSSL} is used to implement S/MIME encryption and
signatures. The system was tested with version 1.0.2h, although any
version of OpenSSL should work that is more recent than 0.9.8n and
1.0.0 [29 Mar 2010], since there have been no changes made to the
S/MIME implementation since then according to the OpenSSL changelog. 

\paragraph{python3} is used as the implementation language. This
implementation was tested with Python3.4 and Python3.5. 

\paragraph{MySQL} is used as the database. We used MySQL version 5.1,
although any modern version should work. 

\paragraph{pymysql} is used to prepare and send the embedded SQL
statements to the MySQL server.

\paragraph{gpg} is used to implement the GPG encryption
algorithm. We tested with version 1.4.16.

\subsection{Test Types}
The database-driven test system can support multiple different
\emph{test types}. Each test type is defined by a sequence of actions
that advance a test instance from test initiation to test
completion. As of September 1, 2016, there are four test types:

\paragraph{bouncer} This test receives an email message and sends a
response. 

\paragraph{dig} This test performs a DNS lookup.

\paragraph{register} This test allows a potential user to register an
email address. 

\paragraph{sendplain} This test sends a plain email message.

\subsection{Database layout}

The database tables are in table~\ref{db}. The 

\begin{table*}
\begin{tabular}{>{\tt}llp{4in}}
%\begin{tabularx}{\textwidth}{>{\tt}llX}
\textrm{table} & database keys & purpose \\
\hline
dns & dnsid, testid & Stores each DNS query and result. \\
keyencodings & keyencodingid & Stores encoding for S/MIME and OpenPGP keys. (Not currently implemented.) \\
messages & messageid, testid & Stores each email message sent or received, including full SMTP log (if available) \\
tests & testid, testtype & Provides a unique \texttt{testid} for each test that is initiated.\\
testtypes & testtype & Provides a unique name for each test type that is implemented by the system.\\
users & userid, email & Maps unique email addresses to unique |userid|s. For each email address tracks a username and the authentication ``hash.''\\
workqueue & workqueueid, testid & Tracks all actions scheduled and completed. Each work item is associated with a specific |testid| and has a set of arguments (|args|), stored as a JSON document.\\
%\end{tabularx}
\end{tabular}
\caption{}\label{db}
\end{table*}


\subsection{File and Class Layout}
The database-driven tester is implemented with an object oriented
famework, with the following classes:

\paragraph{Dbdns} The database DNS implementation. 


\begin{table*}
\begin{tabularx}{\textwidth}{>{\tt}lX}
%\begin{tabular}{ll}
\textrm{file} & purpose\\
\hline
\multicolumn{2}{l}{\textbf{HTML files:}}\\
bootstrap/         & installation of Bootstrap HTML, CSS and JS framework.\\
index.html         & Home page for the tester.\\
jumbotron.css      & Bootstrap theme customization.\\
README.md          & The README program.\\
sendmail.html      & The HTML template used by the \texttt{email.cgi} script.\\
\\
\multicolumn{2}{l}{\textbf{Python Modules and Scripts:}}\\
cron\_runner.py    & Python script run by cron(8) to run the next step in the work queue.  \\
dbdns.py           & The database DNS implementation.\\
dbmaint.py         & The database maintenance program.\\
email\_receiver.py & A script that receives email messages and stores them in the database.\\
Makefile           & Stored commands for testing the test infrastructure.\\
msg-bounce.txt     & A test bounce message used for testing.\\
openpgpkey.py      & An OpenPGPKey implementation, uses \texttt{gnupg} command for cryptography..\\
periodic.py        & The workqueue runner.\\
schema.sql         & The database schema.\\
smimea.py          & An SMIMEA implementation, uses \texttt{openssl} command line for cryptography.\\ 
smtp.py            & An SMTP implementation that records all TCP interactions in the database.\\
tester.py          & The Tester class, which stores test results in the database.\\
\\
\multicolumn{2}{l}{\textbf{CGI Scripts:}}\\
email.cgi         & Allows the user to queue an email message.\\
history.cgi       & Allows users to see their interaction history with the tester.\\
lookup\_email.cgi & Allows users to look up a specific email message\\
lookup\_test.cgi  & Allows users to look up a specific test results.\\
sendmail.cgi      & Allows users to send an email message. \\
webcontrol.cgi    & \\
\end{tabularx}
%\end{tabular}
\caption{Files in the \texttt{server/} directory}\label{tlsa}
\end{table*}

\subsection{System Preparation}
The script |server/centos7-setup.bash| will also install the necessary
Centos modules required to support the database-driven tester.

\subsection{Installation and Operation}
The tester's files should be cloned from the git repository. On
|dane-test| a single git repository is used for both the server tester
and the email tester.

The database can be hosted on the same host as the server or on
another host. If the database is hosted on another Centos instance,
it will be necessary to open up the firewall on the database server
with the commands:

\begin{Verbatim}
$ sudo firewall-cmd --add-service=mysql
$ sudo firewall-cmd --permanent --add-service=mysql
\end{Verbatim}

The testing framework requires a single database instance for its
tables. We recommend having two users that have access to the
instance: a database writer and a database reader. The current system
only supports the database writer user; the reader user will be used
at a later point for statistics.

On the database server, create the database user and the database that
will be used:

\begin{Verbatim}
MariaDB [emaildb]> create database emaildb;
MariaDB [emaildb]> create user 'dbwriter@129.6.100.204' identified by 'dbpassword';
MariaDB [emaildb]> grant all privileges on emaildb.* to 'dbwriter'@'dane-test.had.dnsops.gov';
MariaDB [emaildb]> grant all privileges on emaildb.* to 'dbwriter'@'%';
MariaDB [emaildb]> flush privileges;
\end{Verbatim}

Be sure to test that you can connect to the database from the other
host:

\begin{Verbatim}
[db1]$ mysql -udbwriter -pdbpassword -hdb1.antd.nist.gov emaildb
MariaDB [emaildb]> quit;
[db1]$ 
\end{Verbatim}

Once the database is created, it is necessary to load the schema into
the database. You can do this with the MySQL command:

\begin{Verbatim}
[db1]$ mysql -udbwriter -pdbpassword -hdb1.antd.nist.gov emaildb < email/schema.sql
[db1]$
\end{Verbatim}

Once the database is created and connectivity is verified, the
connection information needs to be stored in the 
|email.cfg| configuration file. The information is stored in this file
so that the username and password do not need to reside in the source
code. \emph{This file should not be stored in the webserver
  hierarchy because it contains the database username, password, and
  host!} Instead, it should be stored in the home directory of the
user that runs the |cron_runner.py| script.

Here is a simple configuration file:

\begin{lstlisting}
[mysql]
username=dbwriter
password=dbpassword
dbname=emaildb
host=localhost
\end{lstlisting}

After the database is created and connectivity is created, a user
needs to be established that will run the scripts. As installed on
|dane-test|, the |slg| user runs the tester.  


Finally, following line needs to be to the user's cron script
(e.g. ``crontab -e''): to run the scripts. This should be done \emph{last!}

\begin{Verbatim}
#
# m h  dom mon dow   command
* * * * * /var/www/html/dane_tester/email/cron_runner.py
\end{Verbatim}


\subsection{Theory of Operation}

\section*{References}

\bibliography{rfc,fy16_report}

\end{document}

% LocalWords:  cryptographic modularization auditability exchanger MX changelog
